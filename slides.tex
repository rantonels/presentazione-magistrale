\pdfminorversion=4
\documentclass[aspectratio=43,mathserif]{beamer}
\usepackage{graphicx,import}
\usepackage{tikz}
\usepackage{calc}
\usepackage{color}

\usepackage{mathtools}  
\mathtoolsset{showonlyrefs}  

% localizzazione
\usepackage[utf8]{inputenc}
\usepackage[T1]{fontenc}
\usepackage[italian]{babel}

\definecolor{yred}{HTML}{d40808}
\definecolor{mblue}{HTML}{003591}



%stile
\usecolortheme{seahorse}
\usefonttheme{default}
\setbeamertemplate{frametitle}[default][center]
\setbeamerfont{frametitle}{size=\LARGE}
\setbeamerfont{title}{size=\huge}

%comandi
\newcommand{\ads}{\ensuremath{\operatorname{AdS}}}
\newcommand{\rfour}{\ensuremath{\mathbb{R}^{1,3}}}
\newcommand{\yfive}{{\color{yred}\ensuremath{Y_5}}}
\newcommand{\ess}{\mathbb{S}}
\newcommand{\ssn}{\mathcal{N}}
\newcommand{\rrep}[1]{\mathbf{#1}}
\newcommand{\cjrep}[1]{\overline{\rrep{#1} }}
\newcommand{\tr}{\operatorname{Tr}}
\newcommand{\vev}[1]{\left\langle{#1}\right\rangle}
\newcommand{\hatt}[1]{\ensuremath{\widehat{#1}}}
\newcommand{\tildd}[1]{\ensuremath{\widetilde{#1}}}
\newcommand{\vol}{\ensuremath{\operatorname{vol}}}
\newcommand{\hodge}{\ensuremath{*}}
\newcommand{\twedge}{\ensuremath{\!\wedge\!}}
\renewcommand{\Re}{\ensuremath{\operatorname{Re}}}
\renewcommand{\Im}{\ensuremath{\operatorname{Im}}}
\newcommand{\leff}{\ensuremath{\mathcal{L}_\text{eff}}}
\newcommand{\yfivetz}{{\color{yred}\ensuremath{Y^{2,0}}}}
\newcommand{\ms}{{\color{mblue}\mathcal{M}}}
\newcommand{\op}{\mathcal{O}}

\title{Teorie Efficaci Olografiche:\\ Un caso di studio}
\author{\vspace{10pt}\\\large Riccardo Antonelli }
\date{\today}

\begin{document}
\begin{frame}
	\maketitle
\end{frame}

\begin{frame}
	\frametitle{QFT fortemente accoppiate}
	Problema fondamentale:
	\begin{center}\textbf{
		Data teoria di campo quantistica fortemente accoppiata, $\longrightarrow$ teoria efficace di bassa energia}
	\end{center}

	\vfill Teoria delle stringhe: equivalenze teorie di gauge 4d $\leftrightarrow$ background di stringa 10d (olografia)

	\vfill \begin{center}Sfruttabili per teoria efficace?\end{center}

	\begin{center} $\downarrow$

		~\\

		\textbf{Teorie Efficaci Olografiche}
	\end{center}



\end{frame}

\begin{frame}
	\frametitle{Superstringhe IIB}
	Teoria di gravità quantistica in 10d.

	\begin{itemize}
		\item Stringhe: oggetti perturbativi 1-dimensionali
		\item D$p$-brane: oggetti non perturbativi $p$-dimensionali; $p$ dispari (D1,D3,D5,\ldots)
	\end{itemize}

	~\\~\\

	A basse energie, le stringhe IIB $\sim$ \textbf{supergravità~IIB} (SUGRA). Teoria di campo, include:

	\begin{itemize}
		\item gravitone $g_{\mu\nu}$, assio-dilatone $\tau$ \quad \quad (${g_s\, = \vev{\Im \tau}^{-1}}$)
		\item $k$-forme: $B_2$, $C_2$, $C_4$ \quad \quad (${\scriptstyle B_2 = \frac{1}{2} B_{\mu\nu} dx^\mu \wedge dx^\nu}$, etc)
		\item + fermioni \ldots
	\end{itemize}
\end{frame}

\begin{frame}
	\frametitle{Olografia}

	Equivalenza fra:

	\begin{itemize}
		\item Teoria di \textbf{gauge} in 4 dimensioni \textbf{conforme}
		\item Teoria delle stringhe IIB (include gravità) su $\ads_5 \times Y^5$
	\end{itemize}

	\begin{figure}[h!]\centering
		\def\svgscale{0.3}
		\import{images/}{ads5y5.pdf_tex}
	\end{figure}

	\begin{columns}
		\begin{column}{0.5\textwidth}
			\begin{flushright}
				$\ads$ (Anti-de Sitter): spaziotempo iperbolico \quad\quad
			\end{flushright}
		\end{column}
		\begin{column}{0.5\textwidth}  %%<--- here
			\quad		\yfive: varietà compatta 5d
		\end{column}
	\end{columns}


\end{frame}

\begin{frame}
	\frametitle{Costruire dualità}
	Si dispongono $N$ D3-brane coincidenti in un background

	\begin{equation}
		\mathbb{R}^{1,3} \times X_6
		\label{}
	\end{equation}


	$X_6$: cono con base $\yfive$: $ds_X^2 = dr^2 + r^2 {\color{yred} ds_Y^2}$

	\begin{figure}[h!]\centering
		\def\svgscale{0.3}
		\import{images/}{branesucono.pdf_tex}
	\end{figure}

\end{frame}

\begin{frame}
	\only<1>{\frametitle{Costruire dualità (2)}}
	\only<2>{\frametitle{Large $N$, strong-coupling} }

	\only<1>{Due visuali equivalenti di questo sistema:}

	\begin{columns}
		\begin{column}{0.5\textwidth}

			\only<2>{\vspace{-20pt}}

	\begin{figure}[h!]\centering
		\def\svgscale{0.25}
		\import{images/}{firstpicture.pdf_tex}
	\end{figure}

\only<1>{	Stringhe aperte attaccate alle D3:
	
	\vfill Teoria di \textbf{gauge} 4d
	
	\vfill $G = SU(N)\times SU(N) \times \ldots$

	\begin{center} $S = -\frac{1}{4}\int d^4 x \tr F^2 + \ldots $ \end{center}


}

\only<2>{
	\begin{center}
	$N \gg 1$,\\
	\vspace{15pt}
	$\lambda := N g_{YM}^2 \gg \infty\,.$\\
	\vspace{15pt}
	Teoria quantistica fort. acc.
\end{center}

}

		\end{column}

	\begin{column}{0.5\textwidth}
	\vspace{-10pt}
	\begin{figure}[h!]\centering
		\def\svgscale{0.4}
		\import{images/}{secondpicture.pdf_tex}
	\end{figure}
	\vspace{-10pt}

	\only<1>{

	Massa D3 curva spaziotempo:
	\vspace{-15pt}
	
			\begin{equation}
				\rfour \times X_6 \longrightarrow \ads_5 \times \yfive\,,
				\label{}
			\end{equation}

			\begin{center}	$ S = 	- \int d^{10}x \sqrt{-g} R + \ldots $ \end{center}

	}

	\only<2>{
		\vspace{-30pt}
	\begin{center}
		$g_s \ll 1$ (``$\hbar \rightarrow 0$\,'')\\
		\vspace{15pt}
		$E \ll 1/l_s$\\
		\vspace{15pt}

		stringhe $\rightarrow$ SUGRA IIB \textbf{classica}
	\end{center}
	}

	\end{column}
\end{columns}
%
%	\begin{enumerate}
%		\vfill\item Stringhe aperte attaccate alle D3 (teoria di gauge in 4d)
%		\vfill\item La massa delle D3 curva lo spaziotempo come
%			\begin{equation}
%				\rfour \times X_6 \longrightarrow \ads_5 \times \yfive\,,
%				\label{}
%			\end{equation}
%
%			$\Rightarrow$ stringhe chiuse su questa geometria warped (teoria 10d)
%	\end{enumerate}
%	\vfill
%	\begin{center}
%		$\Longrightarrow$ dualità olografica
%	\end{center}
\end{frame}
%

%\begin{frame}
%	\frametitle{Large $N$, strong-coupling}
%
%	Si dimostra:
%
%	\vfill
%	\begin{center}
%		
%	Quando in 4d $N\rightarrow \infty$, $\lambda \rightarrow \infty$, 
%	\end{center}
%	
%	\vfill allora in 10d
%
%	\vfill 
%	\begin{center}
%		Stringhe IIB $\rightarrow$ supergravità IIB \textbf{classica} ($\hbar \rightarrow 0$)
%	\end{center}
%	\vfill \[ (\lambda := N^2 g) \]
%\end{frame}




\begin{frame}
	\frametitle{La teoria $Y^{2,0}$}
	Cono $X^{2,0}$ sulla base $\yfivetz \sim \ess^2 \times \ess^3 / \mathbb{Z}_2$

	\vfill 
	\begin{figure}[h!]\centering
		\def\svgscale{0.45}
		\import{images/}{y20geometry.pdf_tex}
	\end{figure}


	$X^{2,0}$ è Calabi-Yau $\Longrightarrow$ teoria superconforme (SCFT) con $\ssn = 1$



	\vfill Supersimmetria \textbf{minimale} (senza la singolarità conica, $\ssn = 4$): teorie meno rigide e più realistiche, dinamica poco studiata

\end{frame}

\begin{frame}
	\frametitle{La teoria $Y^{2,0}$}
	Gruppo di gauge:\vspace{-10pt}

	\begin{equation}
		SU(N)_1\times SU(N)_2 \times SU(N)_3 \times SU(N)_4
		\label{}
	\end{equation}

	\begin{columns}
		\begin{column}{0.75\textwidth}

	Campi di materia: $A_1, A_2, B_1, B_2, C_1, C_2, D_1, D_2$

	\vspace{10pt}

	$A_i \in (\rrep N, \cjrep N, \rrep 1, \rrep 1)$ \quad $B_i \in (\rrep 1, \rrep N , \cjrep N, \rrep 1) $\\
	$C_i \in (\rrep 1, \rrep 1, \rrep N, \cjrep N)$ \quad $D_i \in (\cjrep N, \rrep 1, \rrep 1, \rrep N )$\\

	\vspace{10pt}

	+ superpotenziale (interazione fra i campi):

	\begin{equation}
		W = \lambda \varepsilon^{ij} \varepsilon^{kl} \tr(A_i B_k C_j D_l)
		\label{}
	\end{equation}

		\end{column}
		\begin{column}{0.25\textwidth}
	\vfill Teoria di quiver: 
	\begin{figure}[h!]\centering
		\def\svgscale{0.26}
		\import{images/}{square.pdf_tex}
	\end{figure}


	\end{column}
\end{columns}

\vspace{10pt}
\vfill

	\begin{figure}[h!]\centering
		\def\svgscale{0.45}
		\import{images/}{rgflow.pdf_tex}
	\end{figure}



\end{frame}

\begin{frame}

	Deve esistere una descrizione \textbf{efficace} a bassa energia, in termini di pochi campi dinamici. Come identificarla?


	\begin{figure}[h!]\centering
		\def\svgscale{0.45}
		\import{images/}{modulispace.pdf_tex}
	\end{figure}

	Varietà di vuoti (minimi del potenziale): {\color{mblue}spazio dei moduli $\mathcal{M}$}. Le direzioni lungo $\color{mblue}\mathcal{M}$ sono parametrizzate da \textbf{moduli}. 
	\begin{center}	
		Moduli = campi della teoria efficace
	\end{center}


\end{frame}

\begin{frame}
	\frametitle{$ \ads_5 \times Y^{2,0} $}

	\vspace{-20pt}
	\begin{figure}[h!]\centering
		\def\svgscale{0.5}
		\import{images/}{points.pdf_tex}
	\end{figure}


	\vspace{-30pt}
	
	\vfill

	\begin{center}
		Moduli teoria di campo $\leftrightarrow$ moduli SUGRA IIB	
	\end{center}

	\vfill Teoria classica $\Rightarrow$ è possibile estrarre la dinamica dei moduli

\end{frame}

\begin{frame}
	\frametitle{Moduli SUGRA}
	\vspace{-5pt}
	\begin{figure}[h!]\centering
		\def\svgscale{0.3}
		\import{images/}{moduli.pdf_tex}
	\end{figure}
	\vspace{-10pt}
	\begin{itemize}
		\item Spostare le D3-brane sul cono 
		\item Deformare la metrica (struttura K\"ahler) del cono
		\item Accendere altri campi di SUGRA ($\tau,B_2,C_2,C_4$)
	\end{itemize}
\end{frame}

\begin{frame}

	\vspace{-12pt}

	\begin{figure}[h!]\centering
	\def\svgscale{0.6}
	\import{images/}{sposting.pdf_tex}
	\end{figure}

	\vspace{-20pt}

	\begin{columns}
		\begin{column}{0.48\textwidth}
			\begin{center}	
				$3N$ moduli {\Large $z_I^i$}, (${\scriptstyle i=1,2,3; \,I = 1,\ldots,N}$)
			\end{center}

			\vfill {\centering VEV di \textbf{mesoni}:\par}
	
	\begin{equation}
		\langle\tr \left( A_i B_j C_k D_l \right)\rangle \neq 0
		\label{}
	\end{equation}

		\end{column}
		\begin{column}{0.52\textwidth}

			\vspace{-13pt}
			$2$ moduli $\hatt\rho$, $\tildd\rho$ deformazioni della metrica\\
			
			\vspace{10pt} $1$ modulo $\beta$ per $C_2$, $B_2$


			\vspace{10pt}VEV di \textbf{barioni}:

			\vspace{-13pt}
	\begin{equation}
	\langle	\varepsilon_{a_1 \ldots a_N} \varepsilon^{b_1 \ldots b_N} \underbrace{A^{a_1}_{\;b_1}  \ldots A^{a_N}_{\;b_N} }_{N}\rangle \neq 0
	\end{equation}



		\end{column}\end{columns}



\end{frame}

\begin{frame}
	\begin{itemize}
		\item	$2$ moduli metrica (struttura K\"ahler): la singolarità conica si può ``risolvere'' in due sfere $\ess^2_L \times \ess^2_R$\\
			\vfill
		\item	$1$ modulo per 2-forme $B_2$, $C_2$
	\end{itemize}
	\vspace{-15pt}

	\begin{figure}[h!]\centering
		\def\svgscale{0.5}
		\import{images/}{risoluzione.pdf_tex}
	\end{figure}

	\vspace{-35pt}

	\begin{columns}
		\begin{column}{0.5\textwidth}

			\begin{align}
				\hatt\rho \;\; \sim\;\;  \vol \ess^2_L + \vol \ess^2_R\\
				\tildd\rho \;\; \sim\;\;  \vol \ess^2_L - \vol \ess^2_R
			\end{align}
		\end{column}
		\begin{column}{0.5\textwidth}
			\vfill
			\vspace{10pt}
			\begin{equation}
				\beta \sim \int_{\ess_L^2 + \ess_R^2} B_2 - \tau C_2
				\label{}
			\end{equation}
		\end{column}
	\end{columns}
\end{frame}



\begin{frame}
\begin{tikzpicture}[remember picture,overlay]
	\node[xshift=-3cm,yshift=-2.4cm] at (current page.north east) {
		\def\svgscale{0.4}
		\import{images/}{smallmoduli.pdf_tex}
	};
\end{tikzpicture}
	\frametitle{Teoria efficace}
	$3N+3$ campi $\varphi^m = (z_I^i, \hatt\rho, \tildd\rho, \beta)$\\
	\vfill Calcoliamo $\leff$:
	\vspace{10pt}
	\begin{align}
		\begin{split}
			\leff(\varphi) &= - \pi \mathcal{G}^{ab} \nabla_\mu \rho_a \nabla^\mu \bar\rho_b - 2\pi \sum_{I\in \text{D3}} g_{i\bar\jmath}\, \partial_\mu z_I^i \partial^\mu \bar z_I^{\bar\jmath} - \frac{\pi\mathcal{M}}{\Im \tau} \partial_\mu \beta \,\partial^\mu \bar\beta\\
		&+ \left( \text{fermioni}\ldots \right)
	\end{split}
	\end{align}
	\vspace{-10pt}

	\begin{itemize}
		\item Necessaria $g_{i\bar\jmath}$ esplicita (metrica del cono risolto).
		\vfill \item $\mathcal{G}^{ab},\nabla_\mu,g_{i\bar\jmath},\mathcal{M}$ funzioni complicate di $(\hatt\rho,\tildd\rho,\beta)$ $\Longrightarrow$ forte non-linearità
%		\item $g_{i\bar\jmath}$: metrica (hermitiana) del cono risolto: $\sigma$-model delle D3-brane
%		\item $\leff$ è in realtà la parte bosonica di una Lagrangiana supersimmetrica $\ssn=1$: scalari $\hatt\rho,\tildd\rho,\beta$ accoppiati con superpartner spin-$1/2$.
	\end{itemize}



\end{frame}


\begin{frame}

	\begin{equation}
		g_{2\bar 2} = \chi \,\cosh\left( \frac{1}{3} \cosh^{-1} \left( \chi^{-3} \left( 4t + \frac{\sigma(3\chi^2-\sigma^2)}{2} \right) \right) \right) \, (1+ |y^L|^2)^{-2}
		\label{}
	\end{equation}

	\begin{align}
		t := |\zeta|^2 e^{k^L(y^L) + k^R(y^R)} 
		\label{}
	\end{align}
\end{frame}


\begin{frame}
	\frametitle{Simmetrie}
	Check nontriviale: simmetrie della teoria di campo devono ricomparire nella teoria efficace. 
	\begin{itemize}
		\vfill\item Gruppo superconforme: spontaneamente rotto in generale, verifichiamo l'invarianza di $\leff$ sotto un'implementazione nonlineare.
		\vfill\item La SCFT ha una simmetria di flavour $SU(2)\times SU(2)$. Nella HEFT: è il gruppo di isometria di $\ess^2 \times \ess^2$.
		\vfill\item Tre simmetrie $U(1)$, di cui due anomale. Presenti nella HEFT, le anomale rotte non perturbativamente.
	\end{itemize}
\end{frame}


\begin{frame}
	\frametitle{$U(1)$}

	\begin{itemize}
		\vfill\item \small$U(1)_\text{trace} = U(1)_1 + U(1)_2 + U(1)_3 + U(1)_4$\normalsize \, è disaccoppiato da tutto.
		\vfill\item \small$U(1)_B = U(1)_1 + U(1)_3$\normalsize \, non anomalo. \textbf{Numero barionico}. Nella HEFT:
			\begin{equation}
				\Im \tildd \rho \rightarrow \Im \tildd \rho + \alpha
				\label{}
			\end{equation}

		\vfill\vspace{-2pt}\item Ne rimangono due. Sono: \vspace{-4pt}
			\begin{align}
				U(1)_1 - U(1)_3 &&\leftrightarrow && \Im \hatt \rho \rightarrow \Im \hatt \rho + \alpha\\
				U(1)_4 - U(1)_2 &&\leftrightarrow &&\Im\beta \rightarrow \Im\beta + \alpha
			\end{align}
			Simmetrie classiche della CFT e della $\leff$, ma \textbf{anomale}.\\ Interpretazione olografica: rotte da effetti nonperturbativi $\sim \exp(-N) \sim \exp(-1/g_s)$ dovuti a istantoni di teoria delle stringhe accoppiati ad $\Im\hatt\rho$, $\Im\beta$.

	\end{itemize}
\end{frame}

\begin{frame}
	\begin{center}
		\frametitle{Conclusioni}
		
		Abbiamo dunque costruito la teoria efficace esplicita per un modello \textbf{fortemente accoppiato} e con supersimmetria \textbf{minimale} (intrattabile a livello della teoria di campo).
	\end{center}

		Possibili sviluppi:
\vfill
		\begin{itemize}
			\item Generalizzare $Y^{2,0} \rightarrow Y^{p,q}$
\vfill			\item Espansione perturbativa nei moduli
\vfill	\item Analisi contributi nonperturbativi di stringa
\vfill \item Studio SCFT spontaneamente rotte
		\end{itemize}
\vfill
	\begin{center}
		Grazie per l'attenzione.
	\end{center}
\end{frame}

\end{document}
