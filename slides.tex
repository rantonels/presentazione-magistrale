\pdfminorversion=4
\documentclass[aspectratio=43,mathserif]{beamer}
\usepackage{graphicx,import}
\usepackage{calc}
\usepackage{color}

\usepackage{mathtools}  
\mathtoolsset{showonlyrefs}  

% localizzazione
\usepackage[utf8]{inputenc}
\usepackage[T1]{fontenc}
\usepackage[italian]{babel}

\definecolor{yred}{HTML}{d40808}
\definecolor{mblue}{HTML}{003591}



%stile
\usecolortheme{seahorse}
\usefonttheme{default}
\setbeamertemplate{frametitle}[default][center]
\setbeamerfont{frametitle}{size=\LARGE}
\setbeamerfont{title}{size=\huge}

%comandi
\newcommand{\ads}{\ensuremath{\operatorname{AdS}}}
\newcommand{\rfour}{\ensuremath{\mathbb{R}^{1,3}}}
\newcommand{\yfive}{{\color{yred}\ensuremath{Y_5}}}
\newcommand{\ess}{\mathbb{S}}
\newcommand{\ssn}{\mathcal{N}}
\newcommand{\rrep}[1]{\mathbf{#1}}
\newcommand{\cjrep}[1]{\overline{\rrep{#1} }}
\newcommand{\tr}{\operatorname{Tr}}
\newcommand{\vev}[1]{\left\langle{#1}\right\rangle}
\newcommand{\hatt}[1]{\ensuremath{\widehat{#1}}}
\newcommand{\tildd}[1]{\ensuremath{\widetilde{#1}}}
\newcommand{\vol}{\ensuremath{\operatorname{vol}}}
\newcommand{\hodge}{\ensuremath{*}}
\renewcommand{\Re}{\ensuremath{\operatorname{Re}}}
\renewcommand{\Im}{\ensuremath{\operatorname{Im}}}
\newcommand{\leff}{\ensuremath{\mathcal{L}_\text{eff}}}
\newcommand{\yfivetz}{{\color{yred}\ensuremath{Y^{2,0}}}}

\title{Teorie Efficaci Olografiche:\\ Un caso di studio}
\author{\vspace{10pt}\\\large Riccardo Antonelli }
\date{\today}

\begin{document}
\begin{frame}
	\maketitle
\end{frame}

\begin{frame}
	\frametitle{QFT fortemente accoppiate}
	Problema fondamentale:
	\begin{center}\textbf{
		Data teoria di campo quantistica fortemente accoppiata, $\longrightarrow$ teoria efficace di bassa energia}
	\end{center}

%	\vfill La teoria delle stringhe propone famiglie di equivalenze fra teorie di gauge in 4D e background di stringa in 10D (dualità olografiche) $\rightarrow$ sfruttabili per ricavare la teoria efficace \\
	\vfill Teoria delle stringhe: equivalenze teorie di gauge 4D $\leftrightarrow$ background di stringa 10D (olografia)

	\vfill Sfruttabili per teoria efficace?

	\begin{center} $\downarrow$
	
		~\\
	
	\textbf{teorie efficaci olografiche}
	\end{center}

	

\end{frame}

\begin{frame}
	\frametitle{Superstringhe IIB}
	Teoria di gravità quantistica in 10D.

	\begin{itemize}
		\item Stringhe: oggetti perturbativi 1-dimensionali
		\item D$p$-brane: oggetti non perturbativi $p$-dimensionali; $p$ dispari (D1,D3,D5,\ldots)
	\end{itemize}

	~\\~\\

	A basse energie, le stringhe IIB $\sim$ \textbf{supergravità~IIB} (SUGRA). Teoria di campo, include:

	\begin{itemize}
		\item gravitone $g_{\mu\nu}$, assio-dilatone $\tau$ (complesso)
		\item $k$-forme: $B_2$, $C_2$, $C_4$
		\item + fermioni \ldots
	\end{itemize}
\end{frame}

\begin{frame}
	\frametitle{Olografia}

	Equivalenza fra

	\begin{itemize}
		\item Teoria di gauge in 4 dimensioni
		\item Teoria delle stringhe IIB (include gravità) su $\ads_5 \times Y^5$
	\end{itemize}

	\begin{figure}[h!]\centering
		\def\svgscale{0.3}
	\import{images/}{ads5y5.pdf_tex}
	\end{figure}

	\begin{columns}
	\begin{column}{0.5\textwidth}
	\begin{flushright}
	   $\ads$ (Anti-de Sitter): spaziotempo iperbolico \quad\quad
	\end{flushright}
	\end{column}
	\begin{column}{0.5\textwidth}  %%<--- here
\quad		\yfive: varietà compatta 5D
	\end{column}
	\end{columns}


\end{frame}

\begin{frame}
	\frametitle{Costruire dualità}
	Si dispongono $N$ D3-brane coincidenti in un background

	\begin{equation}
		\mathbb{R}^{1,3} \times X_6
		\label{}
	\end{equation}

	
	$X_6$: cono con base $\yfive$: $ds_X^2 = dr^2 + r^2 {\color{yred} ds_Y^2}$

	\begin{figure}[h!]\centering
		\def\svgscale{0.3}
	\import{images/}{branesucono.pdf_tex}
	\end{figure}

\end{frame}

\begin{frame}
	\frametitle{Costruire dualità (2)}

	Due visuali equivalenti di questo sistema:
	

	\begin{enumerate}
		\vfill\item Stringhe aperte attaccate alle D3 (teoria di gauge in 4D)
		\vfill\item La massa delle D3 curva lo spaziotempo come
			\begin{equation}
				\rfour \times X_6 \longrightarrow \ads_5 \times \yfive\,,
				\label{}
			\end{equation}
			
			$\Rightarrow$ stringhe chiuse su questa geometria warped (teoria 10D)
	\end{enumerate}
\vfill
	\begin{center}
	$1. = 2.$ \quad \quad$\Longrightarrow$ \quad \quad dualità olografica
	\end{center}
\end{frame}
%




\begin{frame}
	\frametitle{La teoria $Y^{2,0}$}
	Cono $X^{2,0}$ sulla base $\yfivetz \sim \ess^2 \times \ess^3 / \mathbb{Z}_2$

	\vfill 
	\begin{figure}[h!]\centering
	\def\svgscale{0.45}
	\import{images/}{y20geometry.pdf_tex}
	\end{figure}


	$X^{2,0}$ è Calabi-Yau $\Longrightarrow$ teoria superconforme (SCFT) con $\ssn = 1$



	\vfill Supersimmetria \textbf{minimale} (senza la singolarità conica, $\ssn = 4$): teorie meno rigide e più realistiche, dinamica pochissimo studiata
	
\end{frame}

\begin{frame}
	\frametitle{La teoria $Y^{2,0}$}
	Gruppo di gauge:\vspace{-10pt}

	\begin{equation}
		U(N)_1\times U(N)_2 \times U(N)_3 \times U(N)_4
		\label{}
	\end{equation}

	\vfill Campi di materia: $A_1, A_2, B_1, B_2, C_1, C_2, D_1, D_2$.

	\vfill$A_i \in (\rrep N, \cjrep N, \rrep 1, \rrep 1)$,

	$B_i \in (\rrep 1, \rrep N , \cjrep N, \rrep 1)$\ldots


	\vfill Teoria di quiver: \vspace{-20pt}

	\begin{figure}[h!]\centering
		\def\svgscale{0.26}
	\import{images/}{square.pdf_tex}
	\end{figure}

	


	+ superpotenziale (interazione fra i campi di materia):

	\begin{equation}
		W = \lambda \varepsilon^{ij} \varepsilon^{kl} \tr(A_i B_k C_j D_l)
		\label{}
	\end{equation}

\end{frame}

\begin{frame}

	Deve esistere una descrizione \textbf{efficace} a bassa energia, in termini di pochi campi dinamici. Come identificarla?


	\begin{figure}[h!]\centering
		\def\svgscale{0.45}
	\import{images/}{modulispace.pdf_tex}
	\end{figure}

	Varietà di vuoti (minimi del potenziale): {\color{mblue}spazio dei moduli $\mathcal{M}$}. Le direzioni lungo $\color{mblue}\mathcal{M}$ sono parametrizzate da \textbf{moduli}. 
	\begin{center}	
		Moduli = campi della teoria efficace!
	\end{center}


\end{frame}

\begin{frame}
	\frametitle{$\ads_5 \times Y^{2,0}$}
	Duale olografico: stringhe IIB sulla geometria

	\begin{equation}
		\ads_5 \times \yfivetz
		\label{}
	\end{equation}

	Quando $N \rightarrow \infty$ e a strong coupling $\longrightarrow$ la string theory diventa la SUGRA IIB classica.\\


	\begin{center}
		
		Moduli della CFT $Y^{2,0}$

	$\updownarrow$

	Moduli di SUGRA su $\ads_5 \times \yfivetz$

	$\updownarrow$

	campi dinamici della teoria efficace

	\end{center}

$\Rightarrow$ è possibile estrarre la Lagrangiana efficace.

\end{frame}

\begin{frame}
	\frametitle{Moduli SUGRA}
	\vspace{-5pt}
	\begin{figure}[h!]\centering
		\def\svgscale{0.3}
	\import{images/}{moduli.pdf_tex}
	\end{figure}
	\vspace{-10pt}
	\begin{itemize}
		\item Spostare le D3-brane sul cono 
		\item Deformare la struttura K\"ahler (metrica) del cono
		\item Accendere altri campi di SUGRA ($\tau,B_2,C_2,C_4$)
	\end{itemize}
\end{frame}

\begin{frame}
	$3N$ moduli immediati:
	\vfill \vspace{-12pt}
	\begin{equation}
		\scalebox{1.5}{$z_I^i\,,$}		\label{}
	\end{equation}
\vfill
	con $(i=1,2,3,\, I = 1,\ldots,N)$: 
	
	\vfill posizioni delle $N$ D3-brane sul cono 6-dimensionale; $3N$ campi complessi.\\


\vfill Sono legati a valori di aspettazione (VEV) di \textbf{mesoni} della CFT.
\end{frame}

\begin{frame}
	Troviamo due moduli della struttura K\"ahler (metrica): la singolarità conica si può ``risolvere'' in due sfere $\ess^2_L \times \ess^2_R$

	~\\

	\begin{figure}[h!]\centering
		\def\svgscale{0.5}
	\import{images/}{risoluzione.pdf_tex}
	\end{figure}

	\vspace{-20pt}

	\begin{align}
		\hatt\rho \;\; \sim\;\;  \vol \ess^2_L + \vol \ess^2_R\\
		\tildd\rho \;\; \sim\;\;  \vol \ess^2_L - \vol \ess^2_R
	\end{align}

	\vfill

	$\Rightarrow$ Due altri campi chirali {\Large $\hatt\rho$, $\tildd\rho$} nella teoria efficace
	
\end{frame}

\begin{frame}
	Troviamo due deformazioni delle 2-forme $B_2$, $C_2$:
\vfill
\vspace{-10pt}
	\begin{align}
		\beta \sim \int_{\ess^2_L + \ess^2_R} \left( B_2 - \tau C_2  \right)\\
		\lambda \sim \int_{\ess^2_L - \ess^2_R} \left(B_2 - \tau C_2\right)
		\label{}
	\end{align}

	\vfill	
	\begin{itemize}
		\item	\scalebox{1.2}{$\beta$}: campo dinamico. 
		\item $\lambda$:  parametro costante, genera una deformazione marginale.
	\end{itemize}

\vfill
\scalebox{1.2}{$\hatt\rho$, $\tildd\rho$, $\beta$} sono altri tre campi chirali nella teoria efficace e corrispondono a VEV di \textbf{barioni}.
\end{frame}


\begin{frame}
	\frametitle{Teoria efficace}
	Ci sono $3N+3$ campi chirali $(z_I^i, \hatt\rho, \tildd\rho, \beta)$ e due parametri marginali $(\tau,\lambda)$. Calcoliamo la $\leff$ efficace:

	\begin{equation}
		\leff = - \pi \mathcal{G}^{ab} \nabla \rho_a \wedge \hodge \nabla \bar\rho_b - 2\pi \sum_I g_{i\bar\jmath} dz^i \wedge \hodge d\bar z^{\bar\jmath} - \frac{\pi\mathcal{M}}{\Im \tau} d\beta \wedge \hodge d\bar \beta
		\label{}
	\end{equation}

	\begin{itemize}
		\item $\mathcal{G}^{ab},\nabla,g_{i\bar\jmath},\mathcal{M}$ sono funzioni complicate di $(\hatt\rho,\tildd\rho,\beta)$ $\Longrightarrow$ forte non-linearità
		\item $g_{i\bar\jmath}$: metrica (hermitiana) del cono risolto: $\sigma$-model delle D3-brane
		\item $\leff$ è in realtà la parte bosonica di una Lagrangiana supersimmetrica $\ssn=1$: gli scalari $\hatt\rho,\tildd\rho,\beta$ sono accoppiati con superpartner spin-$1/2$.
		\item Abbiamo dunque determinato la teoria efficace olografica esatta (per $N = \infty$!)
	\end{itemize}



\end{frame}


\begin{frame}
	\frametitle{Simmetrie}
	Check nontriviale: simmetrie della teoria di campo devono ricomparire nella teoria efficace. 
	\begin{itemize}
		\vfill\item Gruppo superconforme: spontaneamente rotto in generale, verifichiamo l'invarianza di $\mathcal{L}$ sotto un'implementazione nonlineare.
		\vfill\item La SCFT ha una simmetria di flavour $SU(2)\times SU(2)$. Nella HEFT: è il gruppo di isometria di $\ess^2 \times \ess^2$.
		\vfill\item A basse energie il gruppo di gauge si riduce $U(N)^4 \rightarrow SU(N)^4$: cosa succede agli $U(1)$?
	\end{itemize}
\end{frame}


\begin{frame}
	\frametitle{$U(1)$}

	\begin{itemize}
		\vfill\item \small$U(1)_\text{trace} = U(1)_1 + U(1)_2 + U(1)_3 + U(1)_4$\normalsize \, è disaccoppiato da tutto.
		\vfill\item \small$U(1)_B = U(1)_1 + U(1)_3$\normalsize \, non anomalo. \textbf{Numero barionico}. Nella HEFT:
			\begin{equation}
				\Im \tildd \rho \rightarrow \Im \tildd \rho + \alpha
				\label{}
			\end{equation}

		\vfill\vspace{-2pt}\item Ne rimangono due. Sono: \vspace{-4pt}
			\begin{align}
				U(1)_1 - U(1)_3 &&\leftrightarrow && \Im \hatt \rho \rightarrow \Im \hatt \rho + \alpha\\
				U(1)_4 - U(1)_2 &&\leftrightarrow &&\Im\beta \rightarrow \Im\beta + \alpha
			\end{align}
			Simmetrie classiche della CFT e della $\leff$, ma \textbf{anomale}.\\ Interpretazione olografica: rotte da effetti nonperturbativi $\sim \exp(-N) \sim \exp(-1/g_s)$ dovuti a istantoni di teoria delle stringhe accoppiati ad $\Im\hatt\rho$, $\Im\beta$.

	\end{itemize}
\end{frame}

\begin{frame}
	\begin{center}
		Grazie per l'attenzione.
	\end{center}
\end{frame}

%\begin{frame}
%	\frametitle{Visuale di stringhe aperte}
%	Stringhe aperte sul volume dello stack di D3 $\Rightarrow$ teoria 4D
%
%	Per $E \ll (\alpha')^{-1/2}$, modi massless $\longrightarrow$ campi in una QFT.
%	
%
%	~\\
%	
%	Includono:
%
%	\begin{itemize}
%		\item Gluoni di un gruppo di gauge $G$
%		\item Scalari carichi sotto $G$
%	\end{itemize}
%
%	~\\
%
%	$\Rightarrow$ è una teoria di gauge su $R^{1,3}$
%
%\end{frame}
%
%\begin{frame}
%	\frametitle{Visuale di stringhe chiuse}	
%
%	Massa delle D3-brane curva il background in
%
%	\begin{equation}
%		R^{1,3} \times X_6 \longrightarrow \ads_5 \times Y_5
%		\label{}
%	\end{equation}
%
%
%	\begin{itemize}
%		\item $\ads_5$: spaziotempo 5D a curvatura costante negativa. Equivalente Lorentziano di $\mathbb{H}^n$
%		\item $Y_5$: spazio ``interno'' compatto, dipende da $X_6$
%	\end{itemize}
%
%\end{frame}
%\begin{frame}
%	\frametitle{}
%
%	Per $E \ll (\alpha')^{-1/2}$, modi massless $\rightarrow$ campi, includono
%
%	\begin{itemize}
%		\item \textbf{gravitone} $g_{\mu\nu}$, dilatone $\phi$, Kalb-Ramond $B_{2}$
%		\item $k$-forme di Ramond-Ramond: $C_0$, $C_2$, $C_4$
%	\end{itemize}
%
%
%	~\\
%
%	Supergravità di tipo IIB, teoria \textbf{gravitazionale} supersimmetrica in 10D.
%
%	In questo caso, basata sulla geometria warped $\ads_5 \times Y_5$
%
%\end{frame}
%
%\begin{frame}
%	\frametitle{AdS/CFT su coni}
%
%
%\end{frame}
%
%
%
%\begin{frame}
%	\frametitle{Teorie efficaci}
%\end{frame}

\end{document}
